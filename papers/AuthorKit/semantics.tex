%File: formatting-instruction.tex
\documentclass[letterpaper]{article}
\usepackage{aaai}
\usepackage{times}
\usepackage{helvet}
\usepackage{courier}
\frenchspacing
\pdfinfo{
/Title (Using maching learning for automatic semantic feature assignment) %
/Subject (AAAI Publications)
/Author (AAAI Press)}
\setcounter{secnumdepth}{0}  
 \begin{document}
% The file aaai.sty is the style file for AAAI Press 
% proceedings, working notes, and technical reports.
%
\title{}
%\author{}
\maketitle
\begin{abstract}
\begin{quote}
\end{quote}
\end{abstract}


\section{Introduction}

Semantic category assignment or classification of words into semantic fields
is exploited in development of ontologies or various NLP applications such as
word sense disambiguation or question-answering.

In this paper we present our results in experiments with automatically assigning semantic features,
using logistic regression models. The model is trained in a supervised manner
using a small training set of nouns that are annotated with semantic categories.
As features of machine learning \footnote{Initially, in this context of a Machine Learning the notion 'feature'
is used, but we will use 'atribute' instead so that it does not interfere with 'semantic feature' concept}.
we have chosen both morphological and syntactic(context behaviour) properties 
of a noun as features for machine learning.
%The work has been carried out for Czech language.

Recently many tools for automatic semantic annotation have been created, like
clustering words into semantic classes(\cite{baroni:2009} - for German and Italian), 
semantic relation assignment \cite{peirsman}. The most well-known resource
in the field WordNet \cite{wordnet} presents the lexicon of words
interlinked by semantic relations and organized hierarchically into 
semantic classes.
%We have not used WordNet because our study
%is merely of an experimental nature, we plan to use wordnet a resourse for an evaluation.

Our classification according to semantic features is not very fine-grained,
it does not go to such depth like that classification
in WordNet. We work with a minimum set of semantic features (totally 16)
 initially used in the lexicon:
*H - animated, *A - abstract, *C - activity, *R - result , 
*K - concrete, *V - property, *Z - machinery, *INT - 'interval',
 *INS - institution etc. Those categories were selected as 
relevant for a concrete NLP task - an automatical machine translation(MT)
system between Czech and Russian, but they may be also used in other
NLP applications.  

The paper also raises the question of recycling the obsolete language resources and adopting them for the
needs of modern computational linguistics. Another problem is adressed here is whether or not it is 
sufficient to use small  training data in Machine Learning.

The paper is structured as follows. In the first part
we briefly introduce the notion of semantic categories and
show the areas they might be useful for. 
First in Section 2 we describe 
language resources used in the machine learning algorithm. Section 3 
presents the learning algorithms themselves and is divided into two main
parts: experiments with morphological endings as features
and those with features based on context. The evaluation and results are
provided in Section 4. Finally, we conclude in Section 5.

\section{Semantic features - motivation}
Semantic features are generally viewed as components of meaning
that express one definite sense of a word, they are generally associated
with the contrastive context, so they occur either with plus or minus
sign, ex. [+human], [-time] etc.
There is no general set of semantic features, as researches
use their own classification depending on their goals. The
number of semantic features can vary from only a few components (like
devision of nouns into 'animated' or 'non-animated') to the very
fine-grained meaning classification like in WordNet. Semantic features
are therefore closely related to the ontology creating.
They can be also used for some minor research problems, as, for example
resolving nominative-accusative ambiguity in Slavic languages.
In \cite{principled_disam} the authors show the way to disambiguate adjectives
with the help of semantic categories of nouns they modify. For example, 
this discriminates several senses of the  adjective \textit{short}
: applied to those nouns with the sem. feature [+human] versus in combination
with a [+interval] noun. The phrase \textit{short girl} is translated
into Czech as \textit{mala holka}, whereas \textit{short day} is \textit{kratky den}.
This can be also applied to disambiguation of senses of verbs: 
(1) \textit{The dog \bf{runs} after the owner}. - [+human] feature of a subject\\
(2) \textit{The program \bf{runs} on Linux}. -[+computer] feature of a subject \\
The latter verb is translated into Czech with the same verb \it{běžet}
for both senses, so this concrete ambiguity is not relevant for
the translation between English and Czech. On the other hand, in Russian
it might cause a mistake: while the verb in (1) will be still \it{běžat'} in Russian,
the metaphorical meaning in (2) is expressed by another verb - \it{rabotat'} - to work.
More examples on how the semantic features work will be presented in the next section
after the lexicon description.

Recently many tools for automatic semantic annotation have been created, like
clustering words into semantic classes(\cite{wordnet} - for English and other languages,
\cite{baroni:2009} - for German and Italian), semantic relation assignment \cite{peirsman}.

Our classification according to semantic features is not very fine-grained,
it does not go to such depth like that classification
in WordNet. We work with a minimum set of semantic features, like 'animated',
'abstract', 'concrete', 'interval' etc. 

\section{Data sources}
In order to build a model we made use of the following resources:
\begin{itemize}
 \item a Czech-Russian transducing dictionary where words are annotated with semantic features.
 \item monolingual web corpus of Czech texts, lemmatized - 405600 unique lemmas

In order to build a model we made use of the following resources:
\begin{itemize}
 \item a Czech-Russian transducing dictionary where words are annotated with semantic features.
 \item monolingual web corpus of Czech texts, lemmatized - 405600 unique lemmas
\end{itemize}
%First, we will describe the resource that contains information on semantic 
%features - a Czech-Russian dictionary Ruslan.
Semantic data from the first lexicon were used as a small training set,
we also put aside 10\% of it as the test data. 
The machine learning algorithms were applied and both tested on the 
second resource - a monolingual corpus of Czech texts downloaded from the web.

\subsection{Dictionary with semantic features - Ruslan}
The creation of the dictionary \cite{oliva1989parser}
goes back to the 80's when the Rule-Based methods
in Machine Translation prevailed. Generally, the MT systems are constructed for
English and some other language. The choice of Czech and Russian as
a translation pair was partially motivated by a political situation, as

 The project was abandoned in 90's when
there was no need for the MT between Czech and Russian. There were
several attempts to exploit the data and the parts of the system, for
example, in \cite{mt-recycled} authors tried to use the module
of syntactic analysis for Czech for the Czech-English Machine Translation
and \cite{pisa2012} extracted the morphosyntactic information for
valency dictionary.
\subsection{The dictionary entry}
Dictionary entries in Ruslan contain morphological,
syntactic and semantic information.
The dictionary has about 5000 entries, 1080 of which are verbs.
Following is the example of an entry:\\
LE2KAR3==MZ(@(*H),!,MA0111,VRAC2).\\
LE2KAR3 - lékař - lemma, a doctor\\
==MZ - represents part of speech(noun) and declension class.\\
@(*H) - semantic feature 'animated'\\
MA0111,VRAC2 - declination class of Russian lemma + lemma itself\\
%KONZULTOVA==R(5,K,?(N(+(*H,*INS),-(*K,*A,*H),N),A(+(*A,*K),-(*H,*INS),A)
%,/,N(+(*A,*K),-(*H,*INS),N),I(+(*H,*INS),-(*K,*A),I),/,>
%S(I(+(*H,*INS,*Z),S(I)))),91,KONSULTIROVAT6).
We transformed the dictionary into a more readible plaintext format,
both human and machine readible, so that the data can be used in
the machine learning or as a human-readible lexicon:\\
(*A) dan nalog\\
(*C) datování datirovka\\
(*H,*K,*A) děd ded\\
(*A) defekt defekt\\
(*H) moucha mucha\\
(*N) pátek pjatnica\\
There are about 8000 lexical entries in the lexicon, we take 1800
translation equivalents that are nouns with assigned semantic features.
As the resource is quite obsolete(quite often expressions that are not used in Russian
anymore can be found) and the domain of the dictionary is restricted to old manuals
about the mainframe machines, only 2100 nouns
can be found in the modern monolingual corpus.
\subsubsection{Semantic features and how they work in Ruslan}
There are 16 semantic features in Ruslan, and each word can have more than one
feature assigned. In our work we will use only few of them - *H(animated), *A(abstract),
*K(concrete), *INS(institution), *INT(interval). Verbs are assigned by the arguments
which have restriction on nouns with definite semantic properties. In example (1)
the subject of the verb 'to coordinate' can be 'human' or 'institution', whether
the object should have the semantic features either 'concrete' or 'abstract'. The sign (-)
before the set of features shows that those features are prohibited to used in this context.
(1)koordinovat(n(+(*h,*ins,*z,*os)),a(+(*a,*k),-(*h,*ins)),koordinirovat').

Following is the example of how they can influence the disambiguation process.\\
(2cz) Děda nemůže dojít do ordinace sám.\\
lit. Granny couldn't go to hospital on his own\\
(3cz) K této situaci nemůže dojít\\
lit. To this situation couldn't come\\
'Such a situation could not happen'

The ambiguous verb 'dojít' that has a literal sense of 'going somewhere'
and both metaphorical meaning 'to happen' can be translated properly
due to the semantic features assigned to its actants. In (2cz) the
agent has the semantic feature (*H) - animated, so the verb is translated
in its literal meaning(in Russian 'dojti'), whereas in (3cz) the actant does not have this feature
and it is translated as 'proizojti':\\
meaning (1cz) DOJD==R(5,NES,?(N(+(*H,*Z,*P,*F),K(D,DO(G)))),07,DOJTI).\\
meaning (2cz) DOJD==R(5,NES,?(K(D,N)),07,PROIZOJTI)\footnote{We should mention, that in this example
not only the semantics is taken into account, but also the surface realization of an argument}.\\


\subsubsection{Semantic features - negative experience}
As it was stated in \cite{KubonPHD2001} the semantic features did not
served the purpose of disambiguation, sometimes they introduced mistakes
so that translation process failed to produce a result at all. This
happened especially when a valency frame contained the prohibition
to use words with definite semantic features.
We will not use the prohibited features for our prposes, they will be rather "recommended" on the basis of
statistics.


\subsection{Monolingual corpus}
%PDT and Webcol 

\section{Machine Learning}
We realized that the task of semantic category assignment can easily be written as a task of classication, using supervised machine learning algorithms.

Generally, in supervised machine learning, we have a set of training data with defined categories and we want to build an algorithm, which will generalize from this training data and return a category for any yet unseen data.

More specifically, in most of the algorithms, the training data can be broken into \textit{examples}, and each of the examples can be described as a set of \textit{features} and a given \textit{category}.

With our data, we can take the words from the lexicon as training examples and their linguistic categories as categories in machine learning.\footnote{This naming confusion is unfortunate, but it is, we think, unavoidable with mixing machine learning and linguistic semantics.} 

For this approach, we have to solve several problems.

First, in our data, every word in lexicon has more possible semantic categories. Second, in the lexicon alone, we don't see any features for the words.

We solved the first problem by learning a different classifier for every semantic category, where we take the machine learning function as purely binary for every classifier. The machine learning category is 1 when at least one of the semantic categories is a given semantic category, and 0 if none is. For every classifier we use a logistic regression (a discriminative model of machine translation).

The second problem - the machine learning features - are described in the next section.

\subsection{Logistic regression classifier}
(The logistic regression models used are heavily inspired by the Andrew Ng course on Coursera.org Machine Learning MOOC.)

Let's suppose that the feature vector is called $y$ and the category, that we want to predict, is called $x$.

In binary logistic regression model with one cathegory, the classifier learns a vector $\theta$, which is as long as the input feature vector. The vector $\theta$ is then multiplied with the feature vectore $y$, which gives us value $?$. Based on ???, we decide to which category is the given vector classified.

We learn the ???? $?$ using so-called ,,hill climbing'' - for every value $\theta$ we learn the ???? and we try to minimalize it by using ???. There is guaranteed that ????.

\subsection{Features}
\subsubsection{Morphofeatures}
The first feature type we tried were purely morphological features.

We took the beginning of the words - the first letter, the first two letters, three letters and four letters; similarly with the ending. Since endings bear strong semanting meanings in Czech language, we suspected their influence will be high.

Of course, this causes the feature space to explode, which is not wanted; we therefore use just those features, that are seen with at least 1/?? of training data.

\subsubsection{Context features}
Another set of features, that we tried to use, were the word contexts.

For those, we used a separate monolingual corpus, that is described in (...). For every word, we look at all its preceding and following words in the corpus.

Your faithful student,
Twilight Sparkle



\section{Evaluation}

\begin{table}
\caption{Evaluation of ML algorithm wrt different semantic categories}
\begin{center}
\begin{tabular}{ccccccc}\hline
Sem.category &\multicolumn{3}{|c|}{Context} & \multicolumn{3}{|c|}{Morphology}\\
\hline
             & Precision & Recall &F-score                    & Precision& Recall & F\\
\hline
H animated &&& &&&\\
A abstract &&& &&&\\
C activity &&& &&&\\
D action  &&& &&&\\
R result &&& &&&\\
K concrete &&& &&&\\
M measure/unit &&& &&&\\
N instrument &&& &&&\\
V property &&& &&&\\
INT interval &&& &&&\\
INS institution &&& &&&\\
P program &&& &&&\\
F function &&& &&&\\
Z machinery &&& &&&\\

\hline\hline
%Total &&&\\
\end{tabular}
\end{center}
\end{table}


\subsection{Suffixes and a semantic category}
%It is a known fact, that
%In (...) it was stated, that the form of a word can to some extent
%signalyze the semantic class of it. For example, 'ee' - [+human, profession]
Below are the examples of the endings and the respective endings that
can indicate them according to our experiments.
Because of the experiment setup it is rather tricky to trust high precision
score. The latter when it is 1 or approaching it indicates that the 

H animated &&& &&&\\
A abstract : \textit{ivost, ekce, arování, íra, ita, nictví}
%ivost   A       0.333333333333333       0.000220983376023304
%ekce    A       0.469026548672566       0.000532369042237959
%arování A       0.0454545454545455      2.00893978203003e-05
%íra     A       0.929787234042553       0.0043895334237356
%tence   A       0.958990536277603       0.0030535884686856
%ikace   A       0.667664670658683       0.0022399678569634
%ola     A       0.988584474885845       0.0043493546280950
%ita     A       0.550518134715026       0.00426899703681382
%nictví  A       0.295081967213115       0.000542413741148109
%

C activity &&& &&&\\
D action  &&& &&&\\
R result &&& &&&\\
K concrete &&& &&&\\
M measure/unit &&& &&&\\
N instrument &&& &&&\\
V property &&& &&&\\
INT interval &&& &&&\\
INS institution &&& &&&\\
P program &&& &&&\\
F function &&& &&&\\
Z machinery &&& &&&\\


\section{Conclusion}
In this paper we described machine learning algorithms
created to guess the semantic category of a noun based on the following features:\\
1.Context surroundings of a word\\
2.Morphological characteristics of a word\\
We have shown, that for this concrete task the context properties of a word
does not appear to be promising at least in such a simple manner we 
used(regarding context as a word after and a word before) .
Probably, more non-trivial context features should be taken into account 
- such as ex. part-of-speech tag.
On the other hand, such simple method as guessing semantic features on the
basis of the surface form of the word brought more favourable %positive?promising?
results in terms of precision. It was also an interesting observation, that 
adding context features to that morphological to hte machine learning
negatively influnced the overall quality.

The possible future directions of our work will lie both in the field of
optimising features for the Machine Learning algorithm as well as 
adapting WordNet data for our task.
the 

\bibliographystyle{aaai} \bibliography{biblio}


\end{document}
